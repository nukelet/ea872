\documentclass{article}

\usepackage[portuguese]{babel}

\usepackage{amsmath, amssymb}
\usepackage{graphicx}
\usepackage{listings}
\usepackage[colorlinks=true, allcolors=blue]{hyperref}

\usepackage[section]{placeins}

\title{Atividade em sala 11}
\author{Vinícius de Oliveira Peixoto Rodrigues (245294)}
\date{Novembro de 2022}

\begin{document}
\maketitle

\section*{Questão (a)}

Os dois parâmetros são mutuamente exclusivos:

\begin{itemize}
    \item \texttt{rate}: o servidor gera conexões novas com um servidor a uma taxa de \texttt{rate} conexões por segundo
    \item \texttt{parallel}: o servidor tenta manter um dado número de conexões ativas com o servidor
\end{itemize}

\section*{Questão (b)}

\begin{verbatim}
    *** ccuec-parallel.txt ***
    3771 fetches, 5 max parallel, 414810 bytes, in 10 seconds
    110 mean bytes/connection
    377.1 fetches/sec, 41481 bytes/sec
    msecs/connect: 5.87332 mean, 13.013 max, 0.424 min
    msecs/first-response: 6.68218 mean, 15.686 max, 2.926 min
    HTTP response codes:
      code 200 -- 3771
    *** ccuec-rate.txt ***
    49 fetches, 1 max parallel, 5390 bytes, in 10.0248 seconds
    110 mean bytes/connection
    4.8879 fetches/sec, 537.669 bytes/sec
    msecs/connect: 26.0648 mean, 31.769 max, 25.139 min
    msecs/first-response: 29.1671 mean, 33.783 max, 27.645 min
    HTTP response codes:
      code 200 -- 49
    *** ccuec.txt ***
    https://www.ccuec.unicamp.br
    *** fee-parallel.txt ***
    172 fetches, 5 max parallel, 8.11995e+06 bytes, in 10.0032 seconds
    47209 mean bytes/connection
    17.1945 fetches/sec, 811736 bytes/sec
    msecs/connect: 0.311424 mean, 3.263 max, 0.134 min
    msecs/first-response: 270.194 mean, 323.389 max, 237.673 min
    HTTP response codes:
      code 200 -- 172
    *** fee-rate.txt ***
    48 fetches, 3 max parallel, 2.26603e+06 bytes, in 10.0263 seconds
    47209 mean bytes/connection
    4.7874 fetches/sec, 226008 bytes/sec
    msecs/connect: 28.1103 mean, 29.985 max, 25.692 min
    msecs/first-response: 238.871 mean, 270.893 max, 215.767 min
    HTTP response codes:
      code 200 -- 48
    *** fee.txt ***
    https://www.feec.unicamp.br/
\end{verbatim}

Servidores usados:

\begin{itemize}
    \item https://www.feec.unicamp.br/
    \item https://www.ccuec.unicamp.br/
\end{itemize}

Para os testes \texttt{*-parallel.txt}, foi passado o comando

\begin{verbatim}
    ./http_load -parallel 5 -seconds 10 
\end{verbatim}

Para os \texttt{*-rate.txt}:

\begin{verbatim}
    ./http_load -rate 5 -seconds 10 
\end{verbatim}

O endereço IP da máquina usada foi \texttt{143.106.148.123}.

\section*{Questão (c)}

Percebe-se que os servidores conseguem fazer muito mais fetches/s quando os requests são feitos em paralelo, visto que existe um gasto de tempo desnecessário na abertura/fechamento de conexões.

\end{document}
