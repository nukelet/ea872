\documentclass{article}

\usepackage[portuguese]{babel}

\usepackage{amsmath, amssymb}
\usepackage{graphicx}
\usepackage[colorlinks=true, allcolors=blue]{hyperref}

\usepackage[section]{placeins}

\title{Atividade em aula 04}
\author{Vinícius de Oliveira Peixoto Rodrigues (245294)}
\date{Agosto de 2022}

\begin{document}
\maketitle

\section*{Questão 1}
O programa cria um socket, binda em uma porta e escuta/faz \textit{echo} de todas as mensagens recebidas na porta.

\section*{Questão 2}
\begin{itemize}
    \item \texttt{GET}: faz a requisição de um arquivo do servidor, passando o caminho para o arquivo e opcionalmente especificando a versão do protocolo.
    \item \texttt{Accept}: especifica os tipos de arquivos aceitos para requisição
    \item \texttt{Accept-encoding}: os encodings aceitos
    \item \texttt{Accept-language}: as linguagens aceitas
    \item \texttt{Cache-control}: diretivas de cache (max age, max stale, min fresh, etc)
    \item \texttt{Connection}: tipo da conexão (por exemplo, keep-alive)
    \item \texttt{Host}: o host que está fazendo a requisição
    \item \texttt{User-agent}: o agente (por exemplo, navegador) que está enviando a requisição
\end{itemize}

\section*{Questão 3}

\subsection*{Item (a)}

É necessário porque a porta padrão do \texttt{telnet} é a porta 23, e o servidor não aceita conexões HTTP nessa porta.

\subsection*{Item (b)}

A conexão também funciona, porque a porta 443 é a porta padrão para conexões HTTPS.

\section*{Questão 4}

\subsection*{Item (a)}
\begin{verbatim}
    HTTP/1.1 302 Found
    Date: Wed, 14 Sep 2022 18:08:11 GMT
    Server: Apache
    Location: https://www.dca.fee.unicamp.br/
    Content-Length: 215
    Content-Type: text/html; charset=iso-8859-1
    
    <!DOCTYPE HTML PUBLIC "-//IETF//DTD HTML 2.0//EN">
    <html><head>
    <title>302 Found</title>
    </head><body>
    <h1>Found</h1>
    <p>The document has moved <a href="https://www.dca.fee.unicamp.br/">here</a>.</p>
    </body></html>
\end{verbatim}

\subsection*{Item (b)}
\begin{verbatim}
    HTTP/1.1 302 Found
    Date: Wed, 14 Sep 2022 18:30:29 GMT
    Server: Apache
    Location: https://www.dca.fee.unicamp.br/
    Content-Type: text/html; charset=iso-8859-1
\end{verbatim}

Conforme esperado, nós recebemos somente o cabeçalho (o \texttt{HEAD} é idêntico ao \texttt{GET}, exceto que o servidor não deve enviar o corpo da mensagem).

\subsection*{Item (c)}
\begin{verbatim}
    HTTP/1.1 302 Found
    Date: Wed, 14 Sep 2022 18:34:45 GMT
    Server: Apache
    Location: https://www.dca.fee.unicamp.br/index.html
    Content-Length: 225
    Content-Type: text/html; charset=iso-8859-1

    <!DOCTYPE HTML PUBLIC "-//IETF//DTD HTML 2.0//EN">
    <html><head>
    <title>302 Found</title>
    </head><body>
    <h1>Found</h1>
    <p>The document has moved <a href="https://www.dca.fee.unicamp.br/index.html">here</a>.</p>
    </body></html>

\end{verbatim}

Nesse caso não houve diferença, visto que o servidor está redirecionando requisições HTTP para a versão HTTPS do site. 

\subsection*{Item (d)}
\begin{verbatim}
    Host: www.dca.fee.unicamp.brHTTP/1.1 400 Bad Request
    Date: Wed, 14 Sep 2022 18:37:38 GMT
    Server: Apache
    Content-Length: 226
    Connection: close
    Content-Type: text/html; charset=iso-8859-1

    <!DOCTYPE HTML PUBLIC "-//IETF//DTD HTML 2.0//EN">
    <html><head>
    <title>400 Bad Request</title>
    </head><body>
    <h1>Bad Request</h1>
    <p>Your browser sent a request that this server could not understand.<br />
    </p>
    </body></html>
    Connection closed by foreign host.
\end{verbatim}

O servidor retorna um erro "400 - Bad request", visto que faltou o campo indicando o caminho do arquivo requisitado.

\end{document}