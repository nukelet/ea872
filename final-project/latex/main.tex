\documentclass{article}

\usepackage[portuguese]{babel}

\usepackage{amsmath, amssymb}
\usepackage{graphicx}
\usepackage{listings}
\usepackage[colorlinks=true, allcolors=blue]{hyperref}

\usepackage[section]{placeins}

\title{Relatório Final}
\author{Vinícius de Oliveira Peixoto Rodrigues (245294)}
\date{Dezembro de 2022}

\begin{document}
\maketitle

\section*{Descrição do projeto}

\subsection*{Objetivo}
O objetivo do projeto foi construir o esqueleto de um servidor HTTP com os seguintes componentes:

\begin{itemize}
    \item Parser de HTTP
    \item Interface com o sistema de arquivos
    \item Capacidade de atender requisições HTTP simples (GET, HEAD, OPTIONS, TRACE, PUT)
    \item Sistema de autenticação de acesso
    \item Execução de tarefas em paralelo
\end{itemize}

\subsection*{Especificações}

O servidor deve ser capaz de escutar requisições HTTP em uma dada porta e criar threads para atender as requisições. Além disso, ele deve ser capaz de:

\begin{itemize}
    \item Mapear recursos para caminhos reais no sistema de arquivos, tomando como raiz um diretório definido
    \item Realizar controle de acesso por meio de arquivos \texttt{.htaccess}, usando \textit{salt}
\end{itemize}

\end{document}
