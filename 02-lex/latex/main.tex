\documentclass{article}

\usepackage[portuguese]{babel}

\usepackage{amsmath, amssymb}
\usepackage{graphicx}
\usepackage[colorlinks=true, allcolors=blue]{hyperref}

\usepackage[section]{placeins}

\title{Atividade em aula 02}
\author{Vinícius de Oliveira Peixoto Rodrigues (245294)}
\date{Agosto de 2022}

\begin{document}
\maketitle

\section*{Questão 1}
\subsection*{Item (a)}

É necessário colocar o \texttt{./} porque a pasta onde fica o executável gerado não está no \texttt{\$PATH}. Os outros programas (\texttt{gcc}, \texttt{flex}, \texttt{bison}) estão todos em pastas contidas no \texttt{\$PATH} (provavelmente \texttt{/usr/bin} ou \texttt{/usr/local/bin}).

\subsection*{Item (b)}

É possível fazer

\begin{center}
    \texttt{\$ export PATH=\$PWD:\$PATH}
\end{center}

para temporariamente adicionar a pasta local (onde fica o executável) ao \texttt{\$PATH}.

\subsection*{Item (c)}

\begin{enumerate}
    \item O \textit{lexer} gerado dá \textit{match} em letras minúsculas (\texttt{[a-z]})
    \item Em seguida, converte para maíuscula calculando, para um dado caractere \texttt{c}, \texttt{c - ('a' - 'A')} (letras minúsculas tem valor ASCII do que maiúsculas)
\end{enumerate}

Desse modo, o lexer converte todas as minúsculas em um texto para maiúsculas.

\section*{Questão 2}
\subsection*{Item (a)}

\begin{enumerate}
    \item O \textit{lexer} dá \textit{match} em expressões da forma \texttt{"<whitespace> <numero>:"}, onde \texttt{numero} tem dois digitos e fica entre 00 e 23

    \item Agrupa números entre 00 e 11 no símbolo \texttt{AM} e números entre 12 e 23 no símbolo \texttt{PM}
    \item Em seguida, lê "come" o token como um número inteiro e calcula o horário correspondente em AM/PM
\end{enumerate}

\subsection*{Item (b)}

O pipe redireciona o output do primeiro comando (\texttt{date}) para o \texttt{stdin} do programa \texttt{p\_b}

\subsection*{Item (c)}

É possível redirecionar a saída de \texttt{date} para um arquivo, e aí redirecionar o arquivo para o \texttt{stdin} de \texttt{p\_b}:

\begin{center}
    \texttt{\$ date > tmp}

    \texttt{\$ p\_b < tmp}
\end{center}

\end{document}